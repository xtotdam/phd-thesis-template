% !TeX root = phd_thesis_Syromyatnikov.tex

\chapter{Формирование наноструктур на вицинальных поверхностях металлов}
\label{chap:formation}

\blindfootnote{Результаты этой главы были опубликованы в работах A4, A6, A7, A9 из списка публикаций по теме диссертации.}
%\cite{Syromyatnikov:mplb:2016,Syromyatnikov:PhysRevB:2018, SyromyatnikovJetpl2019rus}

\lipsum[1-3]

\section{Раздел 1}
\lipsum[4-7]

\begin{figure}
	\centering
	\tikzsetnextfilename{our_distributions_1T}
	\begin{tikzpicture}
	\pgfplotsset{
		width=0.99\linewidth,
		height=7cm,
		%
		X/.style={mark=none, ultra thick}
	}
	\tikzmath{
		function gauss(\x, \mmu, \ssigma) {
			return exp(- (\x - \mmu)^2. / (2. * \ssigma^2)) / (sqrt(2. * 3.1415) * \ssigma);
		};
	}
	
	\begin{axis}[
	name=curves,
	domain=0:400, samples=401, xmin=0, xmax=400, ymin=0,
	xtick distance=100, ytick distance=0.01,
	yticklabel=\tickprint{fixed,fixed zerofill,precision=2},
	xlabel={Длина, атомы}, ylabel={Вероятность}
	]
	\addplot+ [X,tabred]   { gauss(x, 15, 8) } node[pos=20/400,right] {130 K};
	\addplot+ [X,tabblue]  { gauss(x, 48, 21) } node[pos=48/400,above] {160 K};
	\addplot+ [X,tabgreen] { gauss(x, 115, 53) } node[pos=115/400,above] {190 K};
	\addplot+ [X,taborange]{ gauss(x, 215, 104) } node[pos=215/400,above] {220 K};
	\end{axis}
	
	\tikzset{
		placeholder/.style={minimum width=0.96\linewidth,minimum height=3.6cm,below=0.2cm,draw,help lines},
	}
	
	\node[placeholder] (a130) at (curves.outer south) {};
	\node[placeholder] (a160) at (a130.south) {};
	\node[placeholder] (a190) at (a160.south) {};
	\node[placeholder] (a220) at (a190.south) {};
	
	\tikzmath{\numrows = 20; \ymaxnr = \numrows + 1;};
	
	\pgfplotsset{
		chains/.style = {
			width=1.055\linewidth, height=5.15cm, anchor=center, enlargelimits=0, ymin=0, ymax=\ymaxnr,
			xmax=220,
			%		xmin=-1, xmax=221, x filter/.expression={x > 220 ? nan},
			only marks,	mark size=1pt,
			ticks=none, tick style={draw=none},
			tick label style={overlay}, label style={overlay},
		}
	}
	
	\pgfplotstableread[col sep=comma]{fig/spt/T130.csv}\loadedtableB
	\pgfplotstableread[col sep=comma]{fig/spt/T160.csv}\loadedtableC
	\pgfplotstableread[col sep=comma]{fig/spt/T190.csv}\loadedtableD
	\pgfplotstableread[col sep=comma]{fig/spt/T220.csv}\loadedtableE
	
	\begin{axis}[chains, name=B, at=(a130.center)]
	\foreach \x in {1,...,\numrows}{ \addplot[only marks,tabred] table[x=n\x, y expr={\x}] \loadedtableB; }
	\end{axis}
	
	\begin{axis}[chains, name=C, at=(a160.center)]
	\foreach \x in {1,...,\numrows}{ \addplot[only marks,tabblue] table[x=n\x, y expr={\x}] \loadedtableC; }
	\end{axis}
	
	\begin{axis}[chains, name=D, at=(a190.center)]
	\foreach \x in {1,...,\numrows}{ \addplot[only marks,tabgreen] table[x=n\x, y expr={\x}] \loadedtableD; }
	\end{axis}
	
	\begin{axis}[chains, name=E, at=(a220.center)]
	\foreach \x in {1,...,\numrows}{ \addplot[only marks,taborange] table[x=n\x, y expr={\x}] \loadedtableE; }
	\end{axis}
	
	\node[below left=0.4cm] at (curves.north east) {а};
	
	\tikzset{
		tag/.style={below left,align=right}
	}
	\node[tag] at (B.north west) {б};
	\node[tag] at (C.north west) {в};
	\node[tag] at (D.north west) {г};
	\node[tag] at (E.north west) {д};
	\end{tikzpicture}
	\caption{Распределения длин атомных цепочек (а) и фрагменты массивов цепочек на вицинальной поверхности (б, в, г, д) в системе Co/Cu после напыления при четырех разных температурах: 130~К, 160~K, 190~K, 220~K соответственно. Длина расчетной ячейки составляла 600 атомов, длина представленного фрагмента составляет 200 атомов.}
	\label{fig:our_distributions_1T}
\end{figure}

\section{Раздел 2}
\lipsum[11-14]

\section{Раздел 3}
\lipsum[6-8]
