% !TeX root = phd_thesis_Syromyatnikov.tex

\chapter{ Методы моделирования роста одномерных атомных структур и исследования их структурных свойств}

Для определения атомной структуры на поверхности металлов существует много различных методов. Сюда относятся, в порядке уменьшения точности: ``первопринципные'' методы, метод молекулярной динамики (МД), использующий полуэмпирические потенциалы, методы Монте-Карло. Огромным недостатком ``первопринципных'' методов является их сложность и следующий из неё большой объем вычислений. Метод молекулярной динамики не позволяет исследовать эволюцию систем на больших промежутках времени. Методы Монте-Карло построены на использовании случайных чисел, что позволяет получить статистически точный ответ только по прошествии большого количества шагов.

\begin{figure}
	\centering
	\tikzsetnextfilename{metropolis_scheme}
	\begin{tikzpicture}[scale=3.5]
	\fill[green!30!white] (0,0) -- plot[domain=0:2,smooth,variable=\x] ({\x}, {exp(-\x)}) -- (2,0) -- cycle;
	\fill[red!30!white] (0,1) -- plot[domain=0:2,smooth,variable=\x] ({\x}, {exp(-\x)}) -- (2,1) -- cycle;
	\fill[path fading=west,green!30!white] (0,0) rectangle (-0.5,1);
	\fill[path fading=east,green!30!white] (2,0) -- plot[domain=2:2.2,smooth,variable=\x] ({\x}, {exp(-\x)}) -- (2.2,0) -- cycle;
	\fill[path fading=east,red!30!white] (2,1) -- plot[domain=2:2.2,smooth,variable=\x] ({\x}, {exp(-\x)}) -- (2.2,1) -- cycle;
	
	\draw[-latex] (-0.5,0) -- (2.3,0) node[below right] {$\Delta E$};
	\draw[-latex] (0,0) node[below] {0} -- (0, 1.3) node[below left] {P};
	\node[below left=0 and 0.15] at (0,1) {1};
	\draw (-0.04,1) -- (0.04,1);
	
	\draw[thick] plot[domain=-0.05:2,smooth,variable=\x] ({\x}, {exp(-\x)});
	\node[above right,font=\large] at (1.2,0.3) {$e^{-\frac{\Delta E}{k_BT}}$};
	\end{tikzpicture}
	\caption{Схематическое представление выбора событий в алгоритме Метрополиса. Кривая соответствует функции $\exp \left(-{\Delta E} / {k_BT}\right)$. В зависимости от величины потенциального барьера $\Delta E$, событие либо принимается сразу ($\Delta E<0$) (зеленая зона слева от нуля), либо может быть отброшено с вероятностью $1-P = 1-\exp \left(-{\Delta E} / {k_BT}\right)$ (красная зона над кривой и зеленая зона под кривой справа от нуля). }
	\label{fig:metropolis_scheme}
\end{figure}


\begin{figure}
	\centering
	%	\tikzset{external/force remake}
	\tikzsetnextfilename{kmcChoosing}
	\begin{tikzpicture}
	\tikzmath{\Width = 12;}
	\draw (0,0) rectangle (\Width,1);
	\draw (0.55,0) -- ++(0,1);
	\draw (1.1,0) -- ++(0,1);
	\draw (\Width-0.6,0) -- ++(0,1);
	\node[right] at (0,0.5) {$k_1$};
	\node[right] at (0.55,0.5) {$k_2$};
	\node[left] at (\Width,0.5) {$k_n$};
	
	\tikzmath{\LeftBound = 0.7*\Width; \RightBound = 0.72*\Width;}
	\foreach \x in {1,...,40} \draw ({1.1 + rnd * (\LeftBound - 1.1)}, 0) -- ++(0,1);
	\foreach \x in {1,...,30} \draw ({\RightBound + rnd * (\Width-0.6 - \RightBound)}, 0) -- ++(0,1);
	
	\draw[decorate,decoration={brace,amplitude=5pt,mirror}] (0, -0.1) -- ++(\Width, 0)
	node[midway,below=0.2cm]{$k_{tot}$};
	
	\tikzmath{\Choice = 0.71\Width;}
	\draw[-stealth,very thick] ({1.1 + \Choice * (\Width - 1.1 - 0.6)}, 1.7) -- ++(0,-0.7) node[at start,right] {$r_1 k_{tot}$};
	\draw ({1.1 + \Choice * (\Width - 1.1 - 0.6) - 1}, 1.7) -- ++(1,-1.2) node[at start,left] {$k_q$};
	\end{tikzpicture}
	\caption{Алгоритм случайного выбора состоявшегося события в кинетическом методе Монте-Карло на основании массива частот переходов}
	\label{fig:kmcChoosing}
\end{figure}


\begin{figure}
	\centering
	\tikzset{external/export next=false}
	\begin{tikzpicture}[node distance=0.7cm,bezier bounding box=true]
	\tikzset{
		N/.style={draw,minimum width=4cm,minimum height=1cm,align=left,inner sep=0.5em},
	}
	
	\node[N,shape=chamfered rectangle,minimum width=2cm,fill=red!10] (1) at (0,0) {Начало};
	\node[N,below=of 1] (2) {
		\emph{Инициализация}\\
		Установка параметров расчета\\
		Генерация начального распределения атомов\\
		Распаковка массива потенциальных барьеров
	};
	\node[draw,below=of 2,shape=diamond,align=center,fill=blue!10] (3) {Сделано\\ необходимое\\ число\\ шагов?};
	\node[N,below=1cm of 3] (32) {Построение каталога возможных процессов $\{i \to j\}$};
	\node[N,below=of 32] (4) {Расчет частот $\{k_{ij}\}$~\eqref{eq:transrate}, их суммы $k_{tot}$~\eqref{eq:kmcTotalK}};
	\node[N,below=of 4] (5) {Генерация случайных чисел $r_1$, $r_2$};
	\node[N,below=of 5] (6) {Взвешенный выбор процесса~\eqref{eq:kmcChoose}};
	\node[N,below=of 6] (7) {Реализация выбранного процесса};
	\node[N,below=of 7] (8) {Расчет продолжительности процесса $t_{ij}$~\eqref{eq:kmcHopTime},\\ Обновление таймера};
	\node[N,left=1cm of 3,shape=chamfered rectangle,minimum width=2cm,fill=green!10] (9) {Конец};
	
	\path[->,>=stealth]
	(1) edge (2)
	(2) edge (3)
	(3) edge node[midway,right,red!50!black] {нет} (32)
	(32) edge (4)
	(4) edge (5)
	(5) edge (6)
	(6) edge (7)
	(7) edge (8)
	(8) edge[curve to,out=0,in=0,out looseness=0.5,in looseness=1] (3)
	(3) edge node[midway,above,green!50!black] {да} (9) ;
	\end{tikzpicture}
	\caption{
		Принципиальная схема алгоритма кинетического метода Монте-Карло с априори известными потенциальными барьерами.
	}
	\label{fig:kmcScheme}
\end{figure}
