% !TeX root = phd_thesis_Syromyatnikov.tex

\chapter{Исследование формирования и структурных свойств наноструктур Co на вицинальной поверхности меди (111) (обзор литературы)}

Исследование физики и химии поверхности твердого тела в условиях сверхвысокого вакуума производится в течение уже очень долгого времени~\cite{Giesen2001}. Для этого применяют самые разнообразные методы~\cite{Oura2006}. Например, спектроскопические методы, среди которых спектроскопия характеристических потерь энергии электронами~\cite{Ibach1994,Ibach1991}, фотоэлектронная и инфракрасная спектроскопия~\cite{Greenler1966,Bradshaw1969,DUMAS1999}, которые позволяют получить информацию о химии поверхности и кинетике реакций на ней. Детальное изучение структуры поверхности стало возможным благодаря сканирующей и отражательной электронной микроскопии~\cite{Bauer1994,Bauer1994a}, а также сканирующей туннельной микроскопии~\cite{Binnig1982} (СТМ) и другим сканирующим зондовым методам, таким как атомная силовая микроскопия (АСМ)~\cite{PhysRevLett.56.930}.

Важным элементом поверхности твердого тела являются ее дефекты. Они выступают в роли предпочитаемых мест реакции или прикрепления химически либо физически адсорбированных атомов и молекул. Чтобы иметь представление о кинетике поверхностных дефектов, необходимо иметь информацию о локальном строении поверхности.

Первой вехой в изучении локальной атомной структуры поверхностей и кинетики адатомов на ней в условиях сверхвысокого вакуума стал полевой ионный микроскоп~\cite{Mller1951} (ПИМ), ставший развитием идей, заложенных полевой электронной микроскопией~\cite{Mller1937}. ПИМ способен проводить прямые наблюдения свободных адатомов на острой металлической игле.
Дифракционные методы, среди которых отражательная и низкоэнергетическая электронная микроскопия~\cite{Bauer1994,Bauer1994a}, не ограничены небольшим рядом материалов, в отличие от ПИМ.

Прорывом в изучении локальной атомной структуры поверхности и, в частности, локальных дефектов и их кинетики в ультравысоком вакууме стало применение СТМ\cite{Binnig1982,PhysRevLett.49.57,PhysRevLett.50.120}.
Ранние версии СТМ могли работать только при комнатной температуре, а небольшие колебания температуры во время эксперимента существенно изменяли получаемые данные из-за теплового расширения иглы и остальных компонентов микроскопа.
Появившиеся позже низкотемпературные СТМ~\cite{denHaan2014}, демонстрирующие очень низкий температурный дрейф, можно выделить в отдельный класс экспериментальной аппаратуры. Отдельным преимуществом низкотемпературных исследований является тот факт, что при низкой температуре граница между занятыми и незанятыми электронными состояниями проявляется очень резко, что используется в сверхчувствительной спектроскопии~\cite{Song2010}.
В зависимости от температуры и от изучаемого образца, скорость поверхностной диффузии может быть либо ниже, либо выше скорости получения СТМ изображения. В первом случае отдельные СТМ изображения представляют собой снимки структуры поверхности. Движение атомов и таких дефектов поверхности, как кинки или ступени, может быть изучено путем анализа изменений в последовательных СТМ изображениях. На сегодняшний день высокоскоростной СТМ позволяет наблюдать образец с частотой более 100 кадров в секунду в условиях постоянно меняющейся температуры~\cite{Frenken2017}. Однако во втором случае поверхность изменяется во время записи одного изображения и даже между последовательными линиями сканирования --- изображения несут в себе уже не только пространственную информацию, но и временную. Таким образом, информация о процессах диффузии атомов может быть извлечена только путем обращения к теоретическим моделям, которые способны связать макроскопические изменения в структуре поверхности с отдельными атомными процессами.


\section{Вицинальные поверхности (111) и дефекты на них}

\dots

