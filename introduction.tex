\chapter*{Введение}
\addcontentsline{toc}{chapter}{Введение}
\label{chap:introduction}

\newcommand{\psection}[1]{\vspace*{0.3cm}\textsc{\bfseries{#1}}\par}
\newcommand{\compenspace}{\vspace*{-0.3cm}}

\compenspace

\psection{Актуальность темы}
Научно-технологический прогресс в настоящее время неотделим от постоянной миниатюризации различных электронных устройств, в частности интегральных микросхем и элементов памяти.
Новейшие устройства такого типа строятся на базе наноразмерных электронных компонентов, среди которых транзисторы, цепочки и провода.
Характерные линейные размеры таких объектов сегодня приближаются к единицам нанометров. Оперировать атомными структурами такого размера очень трудно, поэтому прикладные и фундаментальные исследования в этой области сейчас актуальны как никогда.

Помимо этого одномерные наноструктуры обладают рядом уникальных свойств, среди которых волны спиновой и зарядовой плотности~\cite{Erwin2010}, квантованная проводимость~\cite{Klavsyuk2015}, эффект Рашбы~\cite{Rasba84.1}.
Свойства таких одномерных структур существенно отличаются как от свойств объемного образца, так и от свойств тонких пленок и квантовых точек.
Создание наноразмерных структур на текущий момент возможно либо при помощи литографических методов, либо путем перетаскивания отдельных атомов иглой сканирующего туннельного микроскопа.
Оба подхода имеют свои недостатки, не позволяющие получать такие одномерные наноструктуры в промышленных масштабах.
Ситуация может измениться с использованием метода самоорганизации наноструктур путем эпитаксиального роста.
Однако для того чтобы использование наноструктур в промышленности стало реальностью, необходимо уметь управлять их созданием, что невозможно без понимания механизмов, ответственных за их рост.

В связи с этим особенно актуальны исследования структурных, электронных и магнитных свойств наноструктур на поверхности металлов, а также установление закономерностей в их росте и эволюции.


\psection{Цель и задачи работы}
Основной целью диссертационной работы являются исследование механизмов роста атомных структур на поверхности металлов и установление особенностей их атомной структуры с использованием современных теоретических методов. Исследование механизмов роста включает в себя не только моделирование процессов самоорганизации, но и исследование взаимодействия атомов $3d$-металлов на поверхностях металлов. Для достижения поставленной цели решались следующие задачи:
\begin{enumerate}
	\item Разработать методику теоретического исследования и моделирования  формирования и дальнейшей эволюции одномерных атомных структур на поверхности металлов.
	\item Исследовать взаимодействие адатомов $3d$-металлов со ступенью вицинальной поверхности меди.
	\item Определить превалирующие факторы, влияющие на рост одномерных наноструктур на вицинальных металлических поверхностях.
	\item Показать зависимость формы распределения длин атомных цепочек от параметров эксперимента.
	\item Создать модель, описывающую распределение одномерных атомных структур, учитывающую созревание Оствальда и распад коротких одномерных структур.
	\item Исследовать структурный фазовый переход из димеризованного в недимеризованное состояние у атомных цепочек кобальта.
\end{enumerate}

Поставленные задачи важны как с фундаментальной, так и с практической точки зрения.

\psection{Научная новизна работы}

В диссертационной работе получены следующие научные результаты:
\begin{enumerate}
	\item На основе метода Монте-Карло разработана методика численного моделирования формирования и эволюции морфологии одномерных металлических структур на вицинальных металлических поверхностях, а также исследован структурный фазовый переход в них.
	\item Определены особенности взаимодействия адатомов $3d$-металлов со ступенью вицинальной поверхности меди и объяснена их природа. Установлена зависимость взаимодействия между двумя адатомами $3d$-металлов на вицинальной поверхности меди от расстояния до ступени.
	\item Впервые исследован рост и последующая эволюция одномерных атомных структур при двухтемпературном режиме.
	\item Выявлена зависимость формы распределения длин атомных цепочек от таких параметров эксперимента, как температура, поток напыляемых атомов, степень покрытия.
	\item Впервые предложен метод более точного определения значения энергии связи в одномерных атомных структурах.
	\item Установлена зависимость температуры структурного фазового перехода от длины цепочки  и от энергии димеризации.
\end{enumerate}




\psection{Научная и практическая значимость работы}
Полученные результаты могут иметь важное значение в решении прикладных задач по созданию новых методик производства элементов памяти и передачи информации в промышленности. Кроме того, модели описания роста одномерных структур необходимы для более корректного анализа экспериментальных данных.

Работа была  выполнена при финансовой поддержке Российского Фонда Фундаментальных Исследований (гранты 19-32-90045, 19-12-50010) и Фонда развития теоретической физики и математики ``БАЗИС''. При выполнении работы были использованы вычислительные ресурсы Научно-исследовательского вычислительного центра Московского государственного университета им. М.~В.~Ломоносова (НИВЦ МГУ).


\psection{Положения, выносимые на защиту}
\begin{enumerate}
	\item Метод моделирования формирования и эволюции одномерных структур на вицинальных металлических поверхностях.
	\item Микроскопический механизм формирования металлических атомных проводов на вицинальных металлических поверхностях, сценарии эволюции системы атомов при различных внешних условиях.
	\item Модель для описания распределения одномерных атомных структур, в рамках которой возможно более точное определение значения энергии связи.
	\item Структурный фазовый переход в атомных проводах Co на поверхности Cu(775).
\end{enumerate}

\psection{Достоверность результатов}
Результаты, представленные в диссертационной работе, получены с использованием  современных методов теоретической физики. Обоснованность и достоверность определяются адекватностью применяемых моделей и сравнением с экспериментальными данными.

\psection{Апробация результатов}
Основные результаты диссертации были представлены автором лично на следующих мероприятиях:
\begin{itemize}[nosep]
	\item VIII Всероссийская научная молодежная школа-конференция ``Химия, физика, биология: пути интеграции'' (Москва, Россия, 2020);
	\item 8-ая Международная мастерская по магнитным проводам IWMW-2019 (Светлогорск, Россия, 2019);
	\item III Международная Балтийская конференция по магнетизму IBCM-2019 (Светлогорск, Россия, 2019);
	\item Международная конференция по нанонауке и технологии ICN+T (Брно, Чехия, 2018);
	\item VI Научная молодежная школа-конференция ``Химия, физика, биология: пути интеграции'' (Москва, Россия, 2018);
	\item V Научная молодежная школа-конференция ``Химия, физика, биология: пути интеграции'' (с. Ершово, Московская обл., Россия, 2017);
	\item 32-я Европейская конференция физики поверхности ECOSS-32 (Гренобль, Франция, 2016);
	\item 23-я Международная научная конференция студентов, аспирантов и молодых ученых ``Ломоносов-2016'' (Москва, Россия, 2016);
	\item 22-я Международная научная конференция студентов, аспирантов и молодых ученых ``Ломоносов-2015'' (Москва, Россия, 2015);
\end{itemize}

\psection{Публикации}
По теме диссертации опубликовано 10 научных статей и тезисы к 11 докладам на научных конференциях (всего 21 печатная работа).

\psection{Личный вклад автора}
Вклад автора в диссертационную работу является определяющим. Все основные результаты работы получены автором лично, либо при его непосредственном участии.

\psection{Структура и содержание работы}
Диссертация состоит из введения, шести глав, заключения и списка литературы.
Работа изложена на \getrefbykeydefault{LastPage}{page}{0}~страницах, включает \totalfigures~рисунка и \totaltables~таблицы. Общее число ссылок составляет \total{phdthesiscitenum}.
Каждую главу предваряет вступительная часть, представляющая краткое содержание и основные задачи текущей главы.
В конце диссертации сформулированы основные результаты, достигнутые в ней.
